\documentclass[twocolumn]{svjour3}
\usepackage[pdftex]{graphicx}
\usepackage{environ}
\usepackage{amsmath}
\usepackage{amsfonts}
\usepackage[english]{babel}
\usepackage[utf8]{inputenc} 
\usepackage{lmodern}
\usepackage[T1]{fontenc}
\usepackage{color}


% graph drawing
\newcommand{\digraph}[3][scale=1]{ 
  \newwrite\dotfile 
  \immediate\openout\dotfile=dot_temp_#2.dot 
  \immediate\write\dotfile{digraph #2 { margin=0; pad=0; concentrate=false; \string#3}} 
  \immediate\closeout\dotfile
  \immediate\write18{bash -c "dot -Tpdf dot_temp_#2.dot > dot_temp_#2.pdf"}  
  \IfFileExists{dot_temp_#2.pdf}
  % the pdf exists: include it 
  { \includegraphics[#1]{dot_temp_#2} } 
  % the pdf was not created - show a hint
  { \fbox{ \begin{tabular}{l} 
        The file \texttt{dot_temp_#2.pdf} hasn't been created from 
        \texttt{dot_temp_#2.dot} yet. \\
        We attempted to create it with:\\
        `\texttt{dot -Tpdf dot_temp_#2.dot > dot_temp_#2.pdf}' \\
        but that seems not to have worked. You need to execute `\texttt{pdflatex}' with \\
        the `\texttt{-shell-escape} option.
      \end{tabular}} 
  } 
}

\NewEnviron{digraphenv}[2]{\digraph[#1]{#2}{ nodesep="0.1"; ranksep="0.18 equally"; \BODY }}

\newcommand{\control}[2]{#1:successors:s -> #2:predecessors:n [color=red];}
\newcommand{\controllabel}[2]{#1:s -> #2:predecessors:n [color=red];}
\newcommand{\data}[2]{#2:usages:s -> #1:inputs [color=black,dir=back];}
\newcommand{\datalabel}[2]{#2:usages:s -> #1:n [color=black,dir=back];}
\newcommand{\datalabeltext}[3]{#2:usages:s -> #1:n [color=black,dir=back,label="#3"];}

\newcommand{\textnode}[2]{#1 [shape=plaintext, label="#2"]}
\newcommand{\genericnodestart}[1]{#1 [shape=plaintext, label=< <TABLE BORDER="0" CELLSPACING="0"><TR><TD CELLPADDING="0"><TABLE BORDER="0" CELLSPACING="2" CELLPADDING="0"><TR><TD WIDTH="15" HEIGHT="5" PORT="predecessors" BGCOLOR="rosybrown1"></TD></TR></TABLE></TD><TD COLSPAN="2" CELLPADDING="0" ALIGN="RIGHT"><TABLE BORDER="0" CELLSPACING="2" CELLPADDING="0"><TR>}
\newcommand{\genericnodeend}[0]{</TR></TABLE></TD><TD CELLPADDING="0"><TABLE BORDER="0" CELLSPACING="2" CELLPADDING="0"><TR><TD WIDTH="15" HEIGHT="5" PORT="usages" BGCOLOR="lightgrey"></TD></TR></TABLE></TD></TR></TABLE>>]}
\newcommand{\portinput}[1]{<TD WIDTH="15" HEIGHT="5" PORT="#1" BGCOLOR="lightgrey"></TD>}
\newcommand{\portsuccessor}[1]{<TD WIDTH="15" HEIGHT="5" PORT="#1" BGCOLOR="rosybrown1"></TD>}
\newcommand{\portempty}[0]{<TD WIDTH="15" HEIGHT="5"></TD>}
\newcommand{\genericnodelabel}[2]{</TR></TABLE></TD></TR><TR><TD BORDER="1" COLSPAN="3" BGCOLOR="#2"><FONT POINT-SIZE="12">#1</FONT></TD></TR><TR><TD COLSPAN="2" CELLPADDING="0" ALIGN="RIGHT"><TABLE BORDER="0" CELLSPACING="2" CELLPADDING="0"><TR>}

\newcommand{\nodestart}[0]{start [shape=plaintext, rank=source, label=< <TABLE BORDER="0" CELLSPACING="0"><TR><TD COLSPAN="3"><TABLE BORDER="0"><TR><TD></TD> \genericnodelabel{start}{lightyellow} \portsuccessor{successors} \portempty \genericnodeend }

\newcommand{\nodeend}[0]{\genericnodestart{end} \portempty \portinput{inputs} \genericnodelabel{end}{lightyellow} <TD></TD></TR></TABLE></TD><TD></TD></TR></TABLE>>]}

\newcommand{\nodeconst}[2]{#1 [margin="0.05, 0.05", shape=plaintext, label=< <TABLE BORDER="0" CELLSPACING="0"><TR><TD BORDER="1" BGCOLOR="white"><FONT POINT-SIZE="12">#2</FONT></TD></TR><TR><TD WIDTH="15" HEIGHT="5" PORT="usages" BGCOLOR="lightgrey"></TD></TR></TABLE>>]}
\newcommand{\nodeframestate}[2]{#1 [margin="0.05, 0.05", shape=plaintext, label=< <TABLE BORDER="0" CELLSPACING="0"><TR><TD WIDTH="15" HEIGHT="5" PORT="predecessors" BGCOLOR="rosybrown1"></TD><TD WIDTH="15" HEIGHT="5"></TD><TD HEIGHT="5" PORT="inputs" BGCOLOR="lightgrey"></TD></TR><TR><TD BORDER="1" COLSPAN="3" BGCOLOR="lightblue"><FONT POINT-SIZE="12">#2</FONT></TD></TR></TABLE>>]}


\newcommand{\node}[2]{\genericnodestart{#1} \portempty \portinput{inputs} \genericnodelabel{#2}{white} \portsuccessor{successors} \portempty \genericnodeend }
\newcommand{\nodebi}[2]{\genericnodestart{#1} \portinput{in1} \portinput{in2} \genericnodelabel{#2}{white} \portsuccessor{successors} \portempty \genericnodeend }
\newcommand{\nodetri}[2]{\genericnodestart{#1} \portinput{in1} \portinput{in2} \portinput{in3} \genericnodelabel{#2}{white} \portsuccessor{successors} \portempty \portempty \genericnodeend }
\newcommand{\nodequad}[2]{\genericnodestart{#1} \portinput{in1} \portinput{in2} \portinput{in3} \portinput{in4} \genericnodelabel{#2}{white} \portsuccessor{successors} \portempty \portempty \portempty \genericnodeend }
\newcommand{\nodesplit}[2]{\genericnodestart{#1} \portempty \portinput{inputs} \genericnodelabel{#2}{white} \portsuccessor{succ1} \portsuccessor{succ2} \genericnodeend }
\newcommand{\nodequadsplit}[2]{\genericnodestart{#1} \portinput{in1} \portinput{in2} \portinput{in3} \portinput{in4} \genericnodelabel{#2}{white} \portsuccessor{succ1} \portsuccessor{succ2} \portempty \portempty \genericnodeend }
\newcommand{\nodetrisplit}[2]{\genericnodestart{#1} \portinput{in1} \portinput{in2} \portinput{in3} \genericnodelabel{#2}{white} \portsuccessor{succ1} \portsuccessor{succ2} \portempty \genericnodeend }
\newcommand{\nodesplittri}[2]{\genericnodestart{#1} \portempty \portinput{inputs} \genericnodelabel{#2}{white} \portsuccessor{succ1} \portsuccessor{succ2} \portsuccessor{succ3} \genericnodeend }

\newcommand{\nodeguard}[2]{\cnodebi{#1}{#2}{rosybrown1}}

\newcommand{\cnode}[3]{\genericnodestart{#1} \portempty \portinput{inputs} \genericnodelabel{#2}{#3} \portsuccessor{successors} \portempty \genericnodeend }
\newcommand{\cnodebi}[3]{\genericnodestart{#1} \portinput{in1} \portinput{in2} \genericnodelabel{#2}{#3} \portsuccessor{successors} \portempty \genericnodeend }
\newcommand{\cnodetri}[3]{\genericnodestart{#1} \portinput{in1} \portinput{in2} \portinput{in3} \genericnodelabel{#2}{#3} \portsuccessor{successors} \portempty \portempty \genericnodeend }
\newcommand{\cnodesplit}[3]{\genericnodestart{#1} \portempty \portinput{inputs} \genericnodelabel{#2}{#3} \portsuccessor{succ1} \portsuccessor{succ2} \genericnodeend }

% this doesn't work:
%\newenvironment{graphfigure}[2]{\begin{figure}[h] \label{fig:#1} \centering \begin{digraphenv}{scale=0.5}{#1}}{\end{digraphenv} \caption{#2} \end{figure}}

%%%%%%%%%%%%%% example:

% \begin{digraphenv}{scale=0.5}{MyGraph}
% \node{start}{start}
% \node{end}{end}
% % input projections
% \node{a}{proj:a}
% \node{b}{proj:b}
% \data{a}{start}
% \data{b}{start}
% % if
% \nodebi{cmp1}{&lt;}
% \datalabel{cmp1:in1}{a}
% \datalabel{cmp1:in2}{b}
% \nodesplit{if}{if}
% \data{if}{cmp1}
% \control{start}{if}
% % branches
% \nodebi{add1}{+}
% \datalabel{add1:in1}{a}
% \datalabel{add1:in2}{b}
% \nodebi{sub1}{-}
% \datalabel{sub1:in1}{a}
% \datalabel{sub1:in2}{b}
% \controllabel{if:succ1}{add1}
% \controllabel{if:succ2}{sub1}
% % merge
% \nodebi{merge}{merge}
% \control{add1}{merge}
% \control{sub1}{merge}
% % phi
% \nodebi{phi1}{phi}
% \datalabel{phi1:in1}{add1}
% \datalabel{phi1:in2}{sub1}
% % ret
% \node{ret}{ret}
% \data{ret}{phi1}
% \control{merge}{ret}
% \control{ret}{end}
% \end{digraphenv}

%%%%%%%%%%%%%%%


\renewcommand*\descriptionlabel[1]{\hspace\labelsep\normalfont\bf #1}

\newcommand{\Sa}{{\Large$^*$}}
\newcommand{\Sb}{{\Large$^\dag$}}
\newcommand{\Sc}{{\Large$^\S$}}


\newcommand{\mynote}[2]{
\textcolor{red}{\fbox{\bfseries\sffamily\scriptsize#1}
  {\small\textsf{\emph{#2}}}
\fbox{\bfseries\sffamily\scriptsize }}}

\newcommand\TODO[1]{\mynote{TODO}{#1}}
\newcommand\cw[1]{\mynote{CW}{#1}}
\newcommand\ls[1]{\mynote{LS}{#1}}
\newcommand\nodename[1]{\texttt{#1}}



\smartqed  % flush right qed marks, e.g. at end of proof

\journalname{Graal Compiler Design}
\def\makeheadbox{{%
\hbox to0pt{\vbox{\baselineskip=10dd\hrule\hbox
to\hsize{\vrule\kern3pt\vbox{\kern3pt
\hbox{\bfseries The Graal Compiler - Design and Strategy}
\kern3pt}\hfil\kern3pt\vrule}\hrule}%
\hss}}}

\begin{document}

\author{Thomas W\"{u}rthinger \Sa, Lukas Stadler \Sc, Gilles Duboscq \Sa}
\institute{\Sa Oracle, \Sc Johannes Kepler University, Linz}

\date{Created: \today}

\title{The Graal Compiler}
\subtitle{Design and Strategy \\ \textcolor{red}{work in progress (Oracle internal)}}

\maketitle

\abstract{
The Graal compiler (simply referred to as \emph{the compiler} in the rest of this document) aims at improving C1X, the Java port of the HotSpot client compiler, both in terms of modularity and peak performance.
The compiler should work with the Maxine VM and the HotSpot VM.
This document contains information about the proposed design and strategy for developing the compiler.}

\section{Context}

In 2009, the Maxine team started with creating C1X, a Java port of the HotSpot client compiler, and integrate it into the Maxine VM.
Part of this effort was the development of a clear and clean compiler-runtime interface that allows the separation of the compiler and the VM.
This compiler-runtime interface enables the use of one compiler for multiple VMs.
In June 2010, we started integrating C1X into the HotSpot VM and we called the resulting system Graal~VM.
Currently, the Graal~VM is fully functional and runs benchmarks (SciMark, DaCapo) at a similar speed to the HotSpot client compiler.

\section{Goals}
The compiler effort aims at rewriting the high-level intermediate representation of C1X with two main goals:
\begin{description}
\item[Modularity:] A modular design of the compiler should simplify the implementation of new languages, new back-ends, and new optimizations.
\item[Peak Performance:] A more powerful intermediate representation should enable the implementation of heavy-weight optimizations that impact the peak performance of the resulting machine code.
\end{description}

\section{Design}
For the implementation of the compiler, we rely on the following design decisions:
\begin{description}
\item[Graph Representation:]
The compiler's intermediate representation is modeled as a graph with nodes that are connected with directed edges.
There is only a single node base class and every node has an associated graph object that does not change during the node's lifetime.
Every node is serializable and has an id that is unique within its graph.
Every edge is classified as either a control flow edge (anti-dependency) or a data flow edge (dependency) and represented as a simple pointer from the source node to the target node.
It is possible to replace a node with another node without traversing the full graph.
The graph does not allow data flow edge cycles or control flow edge cycles.
We achieve this by explicitely modelling loops (see Section~\ref{sec:loops}). 
\item[Extensibility:]
The compiler is extensible by adding new compiler phases and new node subclasses without modifying the compiler's sources.
A node has an abstract way of expressing its effect and new compiler phases can ask compiler nodes for their properties and capabilities.
We use the ``everything is an extension'' concept.
Even standard compiler optimizations are internally modeled as extensions, to show that the extension mechanism exposes all necessary functionality.
\item[Detailing:]
The compilation starts with a graph that contains nodes that represent the operations of the source language (e.g., one node for an array store to an object array).
During the compilation, the nodes are replaced with more detailed nodes (e.g., the array store node is split into a null check, a bounds check, a store check, and a memory access).
Compiler phases can choose whether they want to work on the earlier versions of the graph (e.g., escape analysis) or on later versions (e.g., null check elimination).
\item[Generality:]
The compiler does not require Java as its input.
This is achieved by having a graph as the starting point of the compilation and not a Java bytecodes array.
Building the graph from the Java bytecodes must happen before giving a method to the compiler.
This enables front-ends for different languages (e.g., Ruby) to provide their own graph.
Also, there is no dependency on a specific back-end, but the output of the compiler is a graph that can then be converted to a different representation in a final compiler phase.
\end{description}

\section{Milestones}
\label{sec:mile}
The compiler is developed starting from the current C1X source code base.
This helps us testing the compiler at every intermediate development step on a variety of Java benchmarks.
We define the following development milestones and when they are considered achieved:
\begin{description}
\item[M1:] We have a fully working Graal~VM version with a stripped down C1X compiler that does not perform any optimizations.
\item[M2:] We modified the high-level intermediate representation to be based on the compiler graph data structure.
\item[M3:] We have reimplemented and reenabled compiler optimizations in the compiler that previously existed in C1X.
\item[M4:] We have reintegrated the new compiler into the Maxine VM and can use it as a Maxine VM bootstrapping compiler.
\end{description}

After those four milestones, we see three different possible further development directions that can be followed in parallel:
\begin{itemize}
  \item Removal of the XIR template mechanism and replacement with a snippet mechanism that works with the compiler graph.
  \item Improvements for peak performance (loop optimizations, escape analysis, bounds check elimination, processing additional interpreter runtime feedback).
  \item Implementation of a prototype front-end for a different language, e.g., JavaScript.
\end{itemize}

\section{Graph}

The \emph{intermediate representation}~(IR) of the compiler is designed as a directed graph.
The graph deals out ids for new nodes and can be queried for the node corresponding to a given id as well as for an unordered list of nodes of the graph.
Graphs can manage side data structures, which will be automatically invalidated and lazily recomputed whenever the graph changes. Examples for side data structures are dominator trees and temporary schedules. These side data structures will usually be understood by more than one optimization.

The nodes of the graph have the following properties:
\begin{itemize}
    \item Each node is always associated with a single graph and this association is immutable.
    \item Each node has an immutable id that is unique within its associated graph.
    \item Nodes can have a data dependency, which means that one node requires the result of another node as its input. The fact that the result of the first node needs to be computed before the second node can be executed introduces a partial order to the set of nodes.
    \item Nodes can have a control flow dependency, which means that the execution of one node will be followed by the execution of another node. This includes conditional execution, memory access serialization and other reasons, and again introduces a partial order to the set of nodes.
    \item Nodes can only have data and control dependencies to nodes which belong to the same graph.
    \item Control dependencies and data dependencies each represent a \emph{directed acyclic graph} (DAG) on the same set of nodes. This means that data dependencies always point upwards, and control dependencies always point downwards in a drawing of the graph. Situations that are normally incur cycles (like loops) are represented by special nodes (see Section~\ref{sec:loops}).
	\item Ordering between nodes is specified only to the extent which is required to correctly express the semantics of a given program. This gives the compiler flexibility for the possible scheduling of a node and therefore wriggle room for optimizations. For algorithms that require a fixed ordering of nodes, a temporary schedule can always be generated.
    \item Both data and control dependencies can be traversed in both directions, so that each node can be traversed in four directions (see Figure~\ref{fig:directions}):
    \begin{itemize}
        \item \emph{inputs} are all nodes that this node has data dependencies on.
        \item \emph{usages} are all nodes whose inputs contain this node.
        \item \emph{successors} are all nodes that have to be after this node in control flow.
        \item \emph{predecessors} are all nodes whose successors contain this node.
    \end{itemize}
    \item Only inputs and successors can be changed, and changes to them will update the usages and predecessors.
    \item Every node must be able to support cloning and serialization.
    \item Inlining should always be performed as embedding one graph into another graph.
    \item Nodes cannot be reassigned to another graph, they are cloned instead.
    \item The edges of a node also define \textit{happens-before} and \textit{happens-after} relationships as shown in Figure~\ref{fig:directions}.
\end{itemize}

\begin{figure}[h]
  \centering
\begin{digraphenv}{scale=0.5}{graphdirections}
\node{node1}{Node}
\textnode{inputs}{inputs}
\textnode{usages}{usages}
\textnode{successors}{successors}
\textnode{predecessors}{predecessors}
\data{node1}{inputs}
\control{node1}{successors}
\data{usages}{node1}
\control{predecessors}{node1}
\node{node2}{Node}
\textnode{before}{happens-before}
\textnode{after}{happens-after}
\data{node2}{before}
\control{node2}{after}
\data{after}{node2}
\control{before}{node2}
\end{digraphenv}
  \caption{A node and its edges.}
  \label{fig:directions}
\end{figure}

\subsection{Project Source Structure}
In order to support the goal of a modular compiler, the code will be divided into the following source code projects (as subprojects of \textbf{com.oracle.graal}).

\begin{description}
    \item[graph] contains the abstract node implementation, the graph implementation and all the associated tools and auxiliary classes.
    \item[nodes] contains the implementation of known basic nodes (e.g., phi nodes, control flow nodes, \ldots).
 				 Additional node classes should go into seperate projects and be specializations of the known basic nodes.
    \item[java] contains code for building graphs from Java bytecodes and Java-specific nodes.
    \item[opt] contains optimizations such as global value numbering or conditional constant propagation.
    \item[compiler] contains the compiler, including:
        \begin{itemize}
            \item Scheduling of the compilation phases.
            \item Implementation of the \emph{compiler interface} (CI).
            \item Implements the final compilation phase that produces the low-level representation.
            \item Machine code creation, including debug info.
        \end{itemize}
\end{description}

\section{Loops}
\label{sec:loops}
Loops form a first-class construct in the IR that is expressed in specialized IR nodes during all optimization phases.
We only compile methods with a control flow where every loop has one single entry point.
This entry point is a \nodename{LoopBegin} node.
This node is connected to a \nodename{LoopEnd} node that merges all control flow paths that do not exit the loop.
The edge between the \nodename{LoopBegin} and the \nodename{LoopEnd} is the backedge of the loop.
It goes from the beginning to the end in order to make the graph acyclic.
An algorithm that traverses the control flow has to explicitely decide whether it wants to incorporate backedges (i.e., special case the treatment of \nodename{LoopEnd}) or ignore them.
Figure \ref{fig:loop1} shows a simple example with a loop with a single entry and two exits.

\begin{figure}[h]
  \centering
\begin{digraphenv}{scale=0.5}{layout1}
\textnode{BeforeLoop}{Loop entry}
\textnode{Exit1}{First loop exit}
\textnode{Exit2}{Second loop exit}
\nodesplit{LoopBegin}{LoopBegin}
\node{LoopEnd}{LoopEnd}
\nodesplit{If1}{If}
\nodesplit{If2}{If}
\controllabel{LoopBegin:succ1}{LoopEnd}
\controllabel{LoopBegin:succ2}{If1}
\controllabel{If1:succ1}{If2}
\controllabel{If2:succ1}{LoopEnd}
\controllabel{BeforeLoop}{LoopBegin}
\controllabel{If1:succ2}{Exit1}
\controllabel{If2:succ2}{Exit2}
\end{digraphenv}
  \caption{A simple loop with two exits.}
  \label{fig:loop1}
\end{figure}

\subsection{Loop Phis}
Data flow in loops is modelled with special phi nodes at the beginning and the end of the loop.
The \nodename{LoopEnd} node merges every value that flows into the next loop iteration in associated \nodename{LoopEndPhi} nodes.
A corresponding \nodename{LoopBeginPhi} node that is associated with the loop header has a control flow dependency on the \nodename{LoopEndPhi} node.
Figure \ref{fig:loop2} shows how a simple counting loop is modelled in the graph.

\begin{figure}[h]
  \centering
\begin{digraphenv}{scale=0.5}{layout2}
\textnode{BeforeLoop}{Loop entry}
\textnode{Exit}{Loop exit}
\textnode{n}{n}
\textnode{Constant0}{0}
\textnode{Constant1}{1}
\nodesplit{LoopBegin}{LoopBegin}
\node{LoopEnd}{LoopEnd}
\nodesplit{If1}{If}
\controllabel{LoopBegin:succ1}{LoopEnd}
\controllabel{LoopBegin:succ2}{If1}
\nodebi{Compare}{&lt;}
\nodebi{LoopBeginPhi}{LoopBeginPhi}
\nodebi{Add}{+}
\datalabel{Add:in1}{LoopBeginPhi}
\datalabel{Add:in2}{Constant1}
\nodebi{LoopEndPhi}{LoopEndPhi}
\control{LoopBeginPhi}{LoopEndPhi}
\data{LoopEndPhi:in1}{LoopEnd}
\data{LoopEndPhi:in2}{Add}
\datalabel{LoopBeginPhi:in1}{LoopBegin}
\datalabel{LoopBeginPhi:in2}{Constant0}
\datalabel{Compare:in1}{LoopBeginPhi}
\datalabel{Compare:in2}{n}
\data{If1}{Compare}
\controllabel{If1:succ1}{LoopEnd}
\controllabel{BeforeLoop}{LoopBegin}
\controllabel{If1:succ2}{Exit}
\end{digraphenv}
  \caption{Graph for a loop counting from 0 to n-1.}
  \label{fig:loop2}
\end{figure}

\subsection{Loop Counters}
The compiler is capable of recognizing variables that are only increased within a loop.
A potential overflow of such a variable is guarded with a trap before the loop.
Figure \ref{fig:loop3} shows the compiler graph of the example loop after the loop counter transformation.


\begin{figure}[h]
  \centering
\begin{digraphenv}{scale=0.5}{layout3}
\textnode{BeforeLoop}{Loop entry}
\textnode{Exit}{Loop exit}
\textnode{n}{n}
\textnode{Constant0}{0}
\textnode{Constant1}{1}
\nodesplit{LoopBegin}{LoopBegin}
\node{LoopEnd}{LoopEnd}
\nodesplit{If1}{If}
\controllabel{LoopBegin:succ1}{LoopEnd}
\controllabel{LoopBegin:succ2}{If1}
\nodebi{Compare}{&lt;}
\nodetri{LoopCounter}{LoopCounter}
\datalabel{LoopCounter:in1}{LoopBegin}
\datalabeltext{LoopCounter:in2}{Constant0}{init}
\datalabeltext{LoopCounter:in3}{Constant1}{stride}
\datalabel{Compare:in1}{LoopCounter}
\datalabel{Compare:in2}{n}
\data{If1}{Compare}
\controllabel{If1:succ1}{LoopEnd}
\controllabel{BeforeLoop}{LoopBegin}
\controllabel{If1:succ2}{Exit}
\end{digraphenv}
  \caption{Graph after loop counter transformation.}
  \label{fig:loop3}
\end{figure}

\subsection{Bounded Loops}

If the total maximum number of iterations of a loop is fixed, then the loop is converted into a bounded loop.
The total number of iterations always denotes the number of full iterations of the loop with the control flowing from the loop begin to the loop end.
If the total number of iterations is reached, the loop is exited directly from the loop header.
In the example, we can infer from the loop exit with the comparison on the loop counter that the total number of iterations of the loop is limited to n.
Figure \ref{fig:loop4} shows the compiler graph of the example loop after the bounded loop transformation.

\begin{figure}[h]
  \centering
\begin{digraphenv}{scale=0.5}{layout4}
\textnode{BeforeLoop}{Loop entry}
\textnode{Exit}{Loop exit}
\textnode{n}{n}
\textnode{Constant0}{0}
\textnode{Constant1}{1}
\nodesplittri{LoopBegin}{BoundedLoopBegin}
\node{LoopEnd}{LoopEnd}
\controllabel{LoopBegin:succ1}{LoopEnd}
\controllabel{LoopBegin:succ2}{LoopEnd}
\controllabel{LoopBegin:succ3}{Exit}
\nodetri{LoopCounter}{LoopCounter}
\datalabel{LoopCounter:in1}{LoopBegin}
\datalabeltext{LoopCounter:in2}{Constant0}{init}
\datalabeltext{LoopCounter:in3}{Constant1}{stride}
\data{LoopBegin}{n}
\controllabel{BeforeLoop}{LoopBegin}
\end{digraphenv}
  \caption{Graph after bounded loop transformation.}
  \label{fig:loop4}
\end{figure}

\subsection{Vectorization}

If we have now a bounded loop with no additional loop exit and no associated phi nodes (only associated loop counters), we can vectorize the loop.
We replace the loop header with a normal instruction that produces a vector of values from 0 to the number of loop iterations minus 1.
The loop counters are replaced with \texttt{VectorAdd} and \texttt{VectorMul} nodes.
The vectorization is only possible if every node of the loop can be replaced with a corresponding vector node.
Figure \ref{fig:loop5} shows the compiler graph of the example loop after vectorization.
The vector nodes all work on an ordered list of integer values and are subject to canonicalization and global value numbering like any other node.


\begin{figure}[h]
  \centering
\begin{digraphenv}{scale=0.5}{layout5}
\textnode{Entry}{Entry}
\textnode{Exit}{Exit}
\textnode{n}{n}
\textnode{Constant0}{0}
\textnode{Constant1}{1}
\node{Vector}{Vector}
\nodebi{VectorAdd}{VectorAdd}
\nodebi{VectorMul}{VectorMul}
\control{Entry}{Vector}
\control{Vector}{Exit}
\datalabel{VectorAdd:in1}{Vector}
\datalabel{VectorAdd:in2}{Constant0}
\datalabel{VectorMul:in1}{VectorAdd}
\datalabel{VectorMul:in2}{Constant1}
\data{Vector}{n}
\end{digraphenv}
  \caption{Graph after vectorization.}
  \label{fig:loop5}
\end{figure}


\section{Frame States}
A frame state captures the state of the program in terms of the Java bytecode specification (i.e., the values of the local variables, the operand stack, and the locked monitors).
Every deoptimization point needs a valid frame state.
A frame state is valid as long as the program performs only actions that can safely be reexecuted.
Thus, frame states need only be generated for the states after bytecodes that cannot be reexecuted:

\begin{itemize}
    \item Array stores: {\tt IASTORE, LASTORE, FASTORE, \\DASTORE, AASTORE, BASTORE, CASTORE, SASTORE}
    \item Field stores: {\tt PUTSTATIC, PUTFIELD}
    \item Method calls: {\tt INVOKEVIRTUAL, INVOKESPECIAL, \\INVOKESTATIC, INVOKEINTERFACE}
    \item Synchronization: {\tt MONITORENTER, MONITOREXIT}
\end{itemize}

Within the graph a frame state is represented as a node that is fixed to the node that caused it to be generated using a control dependency (see Figure~\ref{fig:fs1}).
Frame states also have data dependencies on the contents of the state: the local variables and the expression stack.

The frame state at the method beginning does not have to be explicitely in the graph, because it can always be reconstructed at a later stage.
We save the frame state at control flow merges if there is at least one frame state on any control flow path since the immediate dominator.


\begin{figure}[h]
  \centering
\begin{digraphenv}{scale=0.5}{fs1}
    \nodetrisplit{store1}{ArrayStore}
    \nodebi{load1}{ArrayLoad}
    \controllabel{store1:succ1}{load1}
    \nodetrisplit{store2}{ArrayStore}
    \control{load1}{store2}
    end [shape=plaintext, label="...", width="2.0"]
    store2:succ1:s -> end:n [color=red];
    %
    \nodeframestate{fs1}{FrameState}
    \controllabel{store1:succ2}{fs1}
    \nodeframestate{fs2}{FrameState}
    \controllabel{store2:succ2}{fs2}
\end{digraphenv}
  \caption{Simple example using two frame states.}
  \label{fig:fs1}
\end{figure}


\subsection{Traps}
A trap node is a node that deoptimizes based on a conditional expression.
Trap nodes are not fixed to a certain frame state node, they can move around freely and will always use the correct frame state when the nodes are scheduled (i.e., the last emitted frame state).
The node that is guarded by the deoptimization has a data dependency on the trap, and the trap in turn has a data dependency on the condition and on the most distant node that is postdominated by the guarded node.

In the example of Figure~\ref{fig:trap1}, the second load is guarded by a trap, because its receiver might be null (the receiver of the first load is assumed to be non-null).
The trap is anchored to the control split, because as soon as this node is executed, the second load must be executed as well.
In the final schedule, the trap can be placed before or after the first load.

\begin{figure}[h]
  \centering
\begin{digraphenv}{scale=0.5}{trap1}
    \nodesplit{if}{If}
    \node{split1}{Split}
    \controllabel{if:succ1}{split1}
    nop [shape=plaintext, label="..."]
    \control{split1}{nop}
    %
    \node{split2}{Split}
    \controllabel{if:succ2}{split2}
    \nodebi{load1}{MemLoad}
    \control{split2}{load1}
    \nodebi{load2}{MemLoad}
    \control{load1}{load2}
    \control{load2}{merge}
    \node{merge}{Merge}
    \control{nop}{merge}
    %
    \nodeconst{o1}{obj1}
    \nodeconst{o2}{obj2}
    \datalabel{load1:in2}{o1}
    \datalabel{load2:in2}{o2}
    \nodetrap{trap}{Trap}
    \node{cmpnull}{IsNull}
    \data{cmpnull}{o2}
    \datalabel{trap:in2}{cmpnull}
    \datalabel{load2:in1}{trap}    
    \datalabel{trap:in1}{split2}
\end{digraphenv}
  \caption{A load guarded by a null check trap.}
  \label{fig:trap1}
\end{figure}

Another type of trap is a guard, which is used to remove branches that have a very low execution frequency and replace them with a conditional jump to deoptimization.

In the example of Figure~\ref{fig:trap2}, an \texttt{If} node was replaced by a guard and a trap.
The guard takes the place of the \texttt{If} node in the control flow, and is connected to the trap node.
The trap is now anchored to the most distant node of which the \texttt{If} node was a postdominator.

\begin{figure}[h]
  \centering
\begin{digraphenv}{scale=0.5}{trap2}
    start [shape=plaintext, label="..."]
    start2 [shape=plaintext, label=""]
    start3 [shape=plaintext, label=""]
    \control{start}{guard}
    \node{guard}{Guard}
    \nodebi{load1}{MemLoad}
    \control{guard}{load1}
    \control{load1}{nop}
    nop [shape=plaintext, label="..."]
    %
    \nodetrap{trap}{Trap}
    \datalabel{trap:in1}{start2}
    \datalabel{trap:in2}{start3}
    \data{guard}{trap}    
\end{digraphenv}
  \caption{A trap that is fixed to the control flow using a guard instruction.}
  \label{fig:trap2}
\end{figure}

At some point during the compilation, trap nodes need to be fixed, which means that appropriate data and control dependencies will be inserted so that they cannot move outside the scope of the associated frame state.
This will generate deoptimization-free zones that can be targeted by the most aggressive optimizations.
A simple algorithm for this removal of frame states would be to move all traps as far upwards as possible.
Multiple traps with the same condition and anchor can be merged (see Figure~\ref{fig:trap3}).

\begin{figure}[h]
  \centering
\begin{digraphenv}{scale=0.5}{trap3}
    \nodesplit{if}{If}
    \node{split1}{Split}
    \controllabel{if:succ1}{split1}
    nop [shape=plaintext, label="..."]
    \control{split1}{nop}
    %
    \node{split2}{Split}
    \controllabel{if:succ2}{split2}
    \nodebi{load1}{MemLoad}
    \control{split2}{load1}
    \nodebi{load2}{MemLoad}
    \control{load1}{load2}
    \control{load2}{merge}
    \node{merge}{Merge}
    \control{nop}{merge}
    %
    \nodeconst{o}{obj}
    \datalabel{load1:in2}{o}
    \datalabel{load2:in2}{o}
    \nodetrap{trap}{Trap}
    \node{cmpnull}{IsNull}
    \data{cmpnull}{o}
    \datalabel{trap:in2}{cmpnull}
    \datalabel{load2:in1}{trap}    
    \datalabel{load1:in1}{trap}    
    \datalabel{trap:in1}{split2}
\end{digraphenv}
  \caption{Two loads guarded by the same trap.}
  \label{fig:trap3}
\end{figure}

Also, if two traps that are anchored to the true and false branch of the same \texttt{If} node have the same condition, they can be merged, so that the resulting trap is anchored at the most distant node of which the \texttt{If} node is a postdominator.

\section{Conclusions}
This document sketched the strategy for the Graph compiler.
We already reached M1 (as defined in Section~\ref{sec:mile}) and have the following plans for M2 to M4:
\begin{description}
\item[M2:] June 30th, 2011
\item[M3:] August 15th, 2011
\item[M4:] September 30th, 2011
\end{description}
After we reached M4, we want to create a new project road map that further improves the Graal compiler with respect to its two main goals: Modularity and peak performance.

\end{document}
